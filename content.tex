\hypertarget{ba-186-syllabus}{%
\section{BA 186 Syllabus}\label{ba-186-syllabus}}

\hypertarget{course-details}{%
\section{COURSE DETAILS}\label{course-details}}

BA186 SYSTEMS ANALYSIS AND DESIGN

Principles and methods for analyzing, designing and developing different
types of business and management systems.

\hypertarget{course-description}{%
\subsection{Course Description}\label{course-description}}

This course on Systems Analysis and Design (BA186) teaches methods for
analyzing, designing and assessing different types of business and
management systems. As an introductory Systems Analysis and Design
(SA\&D) course for students in business- or management-related degree
programs, it offers basic knowledge of computerbased business
information systems (I.S.) and systems development practices. It is
designed to give business students an understanding of the impact of
information systems in organizations as well as a better grasp of the
limits and constraints under which I.T. professionals (systems analysts,
programmers, technicians, support staff) work within organizations.

\hypertarget{course-objectives}{%
\subsection{Course Objectives}\label{course-objectives}}

At the end of this course, students must be able to:

\begin{itemize}
\tightlist
\item
  Learn and understand the processes, methodologies and tools involved
  in systems development.
\item
  Recognize which tool and which technique to use in a given phase of
  the systems development process.
\item
  Be exposed to the actual work environment that will provide them the
  opportunity to apply the concepts and principles of systems analysis
  and design.
\item
  Understand the impact of SA\&D activities to support the business
  objectives of organizations
\end{itemize}

\hypertarget{course-methodology}{%
\subsection{Course Methodology}\label{course-methodology}}

This course shall employ a variety of methods such as but not limited
to, lecture-presentation, discussions, teamwork, video presentations,
Q\&A. Individual assignments shall be given and case studies shall be
assigned so as to ascertain the level of understanding of students. The
use of a Learning Management System (LMS) shall be observed all
throughout the course duration. Specifically, e-Learning shall be
employed to acquaint the students of this mode of learning. Submission
of outputs shall be carried out through the LMS. Class interaction is
encouraged using the discussion/chat facilities of the course LMS.

\hypertarget{course-learning-management-system-lms-and-official-class-directory}{%
\subsection{Course Learning Management System (LMS) and Official Class
Directory}\label{course-learning-management-system-lms-and-official-class-directory}}

We will use google classroom. Details will be given on the first day of
class.

You will not receive any email notices if you will not register in both
the LMS and class directory sites. The LMS will provide you access to
course materials, assignments and other information relevant to our
course.

\hypertarget{course-outline}{%
\section{COURSE OUTLINE}\label{course-outline}}

Preliminaries and Creation of Project Teams

Chapter 1: The Systems Development Environment

Chapter 2: The Origins of Software

Presentation of Project Topics by Team Leaders/Introduction of Team
Members

Chapter 3: Managing the Information Systems Project

Chapter 4: Identifying and Selecting Systems Development Projects

Chapter 5: Initiating and Planning Systems Development Projects

Chapter 6: Determining System Requirements Fieldwork/Team Work/Lab Work

\textbf{Midterm Examination}

Chapter 7: Structuring System Process Requirements Fieldwork/Team
Work/Lab Work

Chapter 8: Structuring System Data Requirements Fieldwork/Team Work/Lab
Work

Chapter 9: Designing Databases Fieldwork/Team Work/Lab Work

Chapter 10: Designing Forms and Reports Fieldwork/Team Work/Lab Work

Chapter 11: Designing Interfaces and Dialogues Fieldwork/Team Work/Lab
Work

Chapter 12: Designing Distributed and Internet Systems

\textbf{Pre-Final Examination}

Team Presentation

Team Presentation

Team Presentation

Team Presentation

Submission of Final Deliverables

Final Examination*

Note: \emph{Based on actual circumstances and assessing the progress of
the class, the Instructor reserves the right to change/
collapse/interchange any of the topics as listed above.}

\hypertarget{course-requirements}{%
\section{COURSE REQUIREMENTS}\label{course-requirements}}

The class shall observe the following course deliverables:

\begin{enumerate}
\def\labelenumi{\arabic{enumi}.}
\tightlist
\item
  Written Exams (Midterm and Pre-Final)
\item
  Final Exam (for those who will not meet the average score of 60\%,
  midterm and pre-final exams)
\item
  Quizzes (shall be given at the discretion of the Instructor)
\item
  Team Presentations and Deliverables
\end{enumerate}

At the end of the semester, each student shall be evaluated based on the
following items:

\begin{longtable}[]{@{}ll@{}}
\toprule
Requirement & Weight\tabularnewline
\midrule
\endhead
Class Participation/Attendance/Individual Assignments &
10\%\tabularnewline
Midterm Exam and Pre-Final Exam & 30\%\tabularnewline
Group Tasks/Assignments & 10\%\tabularnewline
SA\&D Deliverables & 40\%\tabularnewline
Team Presentation & 10\%\tabularnewline
\textbf{Total} & \textbf{100\%}\tabularnewline
\bottomrule
\end{longtable}

\hypertarget{grade-equivalents}{%
\section{GRADE EQUIVALENTS}\label{grade-equivalents}}

The scores of the students shall be matched using this grading scale:

\begin{longtable}[]{@{}ll@{}}
\toprule
SCORE FINAL & GRADE\tabularnewline
\midrule
\endhead
92 -- 100 & 1.0\tabularnewline
88 - below 92 & 1.25\tabularnewline
85 - below 88 & 1.5\tabularnewline
82 - below 85 & 1.75\tabularnewline
78 - below 82 & 2.0\tabularnewline
74 - below 78 & 2.25\tabularnewline
70 - below 74 & 2.5\tabularnewline
65 - below 70 & 2.75\tabularnewline
60 - below 65 & 3.0\tabularnewline
below 60 & 5.0\tabularnewline
\bottomrule
\end{longtable}

\hypertarget{course-policies}{%
\section{COURSE POLICIES}\label{course-policies}}

\hypertarget{project-teams}{%
\subsection{Project Teams}\label{project-teams}}

The class will be grouped into teams. Each team will have at most five
(5) members. These teams will work on the SA\&D deliverables, group
assignments and in completing all the other requirements of the course.

\hypertarget{attendance}{%
\subsection{Attendance}\label{attendance}}

The class shall observe the University rules on attendance. This applies
to sessions where physical presence is required. Four (4) absences
(excused/unexcused) are allowed all throughout the duration of the
course. A student who exceeds this maximum allowable absences shall be
dropped from the class roll. The Instructor shall check the attendance
from time to time and this shall be done on a random basis.

\hypertarget{submission-of-requirements}{%
\subsection{Submission of
Requirements}\label{submission-of-requirements}}

Students are expected to demonstrate utmost diligence in the submission
of course requirements. For every day of delay (includes weekends and
holidays), five (5) points shall be deducted from the overall score
awarded to any work/task. Time management should be observed by students
so as to ensure prompt and quality submission of the course
requirements.

\hypertarget{code-of-ethics}{%
\subsection{Code of Ethics}\label{code-of-ethics}}

The highest level of ethical standards must be observed by each and
every class member. Being admitted into the program is an accomplishment
that should not be tarnished with any pigment of fraud, deceit and
distrust. The course will be delivered under the clear cloud of mutual
trust and respect.

\hypertarget{peer-evaluation}{%
\subsection{Peer Evaluation}\label{peer-evaluation}}

Each member of the team will accomplish an online evaluation form at the
end of the course. This evaluation form will capture each member's
assessment of the performance of his/her team members. A rating of 1
shall be given to team members whose performance is Very Poor, 3 for
Average performance and 5 for those who have Excellent performance. For
each missing point, two (2) points shall be deducted from the team score
and the student shall receive the resulting score (Illustration: if team
score is 90 and peer evaluation rating of student A is 3, student A will
receive a score of 86). Thus, it is important that the peer evaluation
will be done with utmost care and consideration of what really was
delivered and performed by each member of the team.

\hypertarget{group-tasksassignments}{%
\subsection{Group Tasks/Assignments}\label{group-tasksassignments}}

The course will require groups/teams to work on various tasks and
assignments. These group tasks/assignments shall be assigned to the
teams at a particular time and submission deadlines shall be specified
by the Instructor for each of the tasks/assignments. Papers related to
the this team work must be uploaded into the LMS. The evaluation of
papers will follow these metrics:

\begin{longtable}[]{@{}ll@{}}
\toprule
RUBRIC & \%\tabularnewline
\midrule
\endhead
Organization and Logic & 30\%\tabularnewline
Depth of Analysis & 40\%\tabularnewline
Presentation/Delivery of Ideas & 20\%\tabularnewline
Overall Impact of Paper & 10\%\tabularnewline
\bottomrule
\end{longtable}

\hypertarget{other-policies}{%
\subsection{Other Policies}\label{other-policies}}

\begin{enumerate}
\def\labelenumi{\arabic{enumi}.}
\tightlist
\item
  Complaints regarding exam results shall be entertained only within a
  period of one week after the examination papers are returned. No
  complaints shall be entertained after this period. Students must use a
  pen or ballpen when taking any examination, otherwise, no complaints
  whatsoever regarding the examination shall be accepted.
\item
  The specific guidelines that shall govern the other course
  requirements (team presentations and deliverables) shall be announced
  in the class prior to the implementation of said course requirements.
  Each day of delay (including Saturdays and Sundays) shall be penalized
  by deducting 5 points from whatever score earned by a student/team for
  a specific requirement.
\item
  Students who will get an average score of 60\% or higher shall be
  exempted from taking the Final Exam. A student who will miss one of
  the exams (i.e.~midterm or pre-final) shall receive 80\% of the score
  of the other exam taken by the student. If he/she will miss both
  exams, the student will take a Completion Exam and the Final Exam
  (Thursday, Oct 10). Those who will take the Completion and Final Exam
  must expect a set of more difficult questions.
\item
  Teams are expected to submit ALL deliverables. If one item is missing,
  all team members shall automatically receive a grade of INC. For
  completing said requirements, item \#3 above shall be applied in the
  computation of scores.
\end{enumerate}

\hypertarget{course-references}{%
\section{COURSE REFERENCES}\label{course-references}}

\begin{itemize}
\tightlist
\item
  Modern Systems Analysis and Design, 6th edition. Jeffrey A. Hoffer,
  Joey F. George, and Joseph S. Valacich,2011.
\item
  Systems Analysis and Design. Kenneth E. Kendall and Julie E. Kendall,
  6th edition.
\end{itemize}
